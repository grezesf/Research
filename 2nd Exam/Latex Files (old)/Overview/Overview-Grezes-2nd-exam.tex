%%%%%%%%%%%%%%%%%%%%%%%%%%%%%%%%%%%%%%%%%
% Arsclassica Article
% LaTeX Template
% Version 1.1 (10/6/14)
%
% This template has been downloaded from:
% http://www.LaTeXTemplates.com
%
% Original author:
% Lorenzo Pantieri (http://www.lorenzopantieri.net) with extensive modifications by:
% Vel (vel@latextemplates.com)
%
% License:
% CC BY-NC-SA 3.0 (http://creativecommons.org/licenses/by-nc-sa/3.0/)
%
%%%%%%%%%%%%%%%%%%%%%%%%%%%%%%%%%%%%%%%%%

%----------------------------------------------------------------------------------------
%	PACKAGES AND OTHER DOCUMENT CONFIGURATIONS
%----------------------------------------------------------------------------------------

\documentclass[
10pt, % Main document font size
a4paper, % Paper type, use 'letterpaper' for US Letter paper
oneside, % One page layout (no page indentation)
%twoside, % Two page layout (page indentation for binding and different headers)
headinclude,footinclude, % Extra spacing for the header and footer
BCOR5mm, % Binding correction
]{scrartcl}

%%%%%%%%%%%%%%%%%%%%%%%%%%%%%%%%%%%%%%%%%
% Arsclassica Article
% Structure Specification File
%
% This file has been downloaded from:
% http://www.LaTeXTemplates.com
%
% Original author:
% Lorenzo Pantieri (http://www.lorenzopantieri.net) with extensive modifications by:
% Vel (vel@latextemplates.com)
%
% License:
% CC BY-NC-SA 3.0 (http://creativecommons.org/licenses/by-nc-sa/3.0/)
%
%%%%%%%%%%%%%%%%%%%%%%%%%%%%%%%%%%%%%%%%%

%----------------------------------------------------------------------------------------
%	REQUIRED PACKAGES
%----------------------------------------------------------------------------------------

\usepackage[
nochapters, % Turn off chapters since this is an article        
beramono, % Use the Bera Mono font for monospaced text (\texttt)
eulermath,% Use the Euler font for mathematics
pdfspacing, % Makes use of pdftex’ letter spacing capabilities via the microtype package
dottedtoc % Dotted lines leading to the page numbers in the table of contents
]{classicthesis} % The layout is based on the Classic Thesis style

\usepackage{arsclassica} % Modifies the Classic Thesis package

\usepackage[T1]{fontenc} % Use 8-bit encoding that has 256 glyphs

\usepackage[utf8]{inputenc} % Required for including letters with accents

\usepackage{graphicx} % Required for including images
\graphicspath{{Figures/}} % Set the default folder for images

\usepackage{enumitem} % Required for manipulating the whitespace between and within lists

\usepackage{lipsum} % Used for inserting dummy 'Lorem ipsum' text into the template

\usepackage{subfig} % Required for creating figures with multiple parts (subfigures)

\usepackage{amsmath,amssymb,amsthm} % For including math equations, theorems, symbols, etc

\usepackage{varioref} % More descriptive referencing

%----------------------------------------------------------------------------------------
%	THEOREM STYLES
%---------------------------------------------------------------------------------------

\theoremstyle{definition} % Define theorem styles here based on the definition style (used for definitions and examples)
\newtheorem{definition}{Definition}

\theoremstyle{plain} % Define theorem styles here based on the plain style (used for theorems, lemmas, propositions)
\newtheorem{theorem}{Theorem}

\theoremstyle{remark} % Define theorem styles here based on the remark style (used for remarks and notes)

%----------------------------------------------------------------------------------------
%	HYPERLINKS
%---------------------------------------------------------------------------------------

\hypersetup{
%draft, % Uncomment to remove all links (useful for printing in black and white)
colorlinks=true, breaklinks=true, bookmarks=true,bookmarksnumbered,
urlcolor=webbrown, linkcolor=RoyalBlue, citecolor=webgreen, % Link colors
pdftitle={}, % PDF title
pdfauthor={\textcopyright}, % PDF Author
pdfsubject={}, % PDF Subject
pdfkeywords={}, % PDF Keywords
pdfcreator={pdfLaTeX}, % PDF Creator
pdfproducer={LaTeX with hyperref and ClassicThesis} % PDF producer
} % Include the structure.tex file which specified the document structure and layout

\hyphenation{Fortran hy-phen-ation} % Specify custom hyphenation points in words with dashes where you would like hyphenation to occur, or alternatively, don't put any dashes in a word to stop hyphenation altogether

%----------------------------------------------------------------------------------------
%	TITLE AND AUTHOR(S)
%----------------------------------------------------------------------------------------

\title{\normalfont\spacedallcaps{Reservoir Computing}} % The article title

\author{\spacedlowsmallcaps{by Felix Grezes*}} % The article author(s) - author affiliations need to be specified in the AUTHOR AFFILIATIONS block

\date{2014} % An optional date to appear under the author(s)

%----------------------------------------------------------------------------------------

\begin{document}

%----------------------------------------------------------------------------------------
%	HEADERS
%----------------------------------------------------------------------------------------

\renewcommand{\sectionmark}[1]{\markright{\spacedlowsmallcaps{#1}}} % The header for all pages (oneside) or for even pages (twoside)
%\renewcommand{\subsectionmark}[1]{\markright{\thesubsection~#1}} % Uncomment when using the twoside option - this modifies the header on odd pages
\lehead{\mbox{\llap{\small\thepage\kern1em\color{halfgray} \vline}\color{halfgray}\hspace{0.5em}\rightmark\hfil}} % The header style

\pagestyle{scrheadings} % Enable the headers specified in this block

%----------------------------------------------------------------------------------------
%	TABLE OF CONTENTS & LISTS OF FIGURES AND TABLES
%----------------------------------------------------------------------------------------

\maketitle % Print the title/author/date block

\setcounter{tocdepth}{2} % Set the depth of the table of contents to show sections and subsections only

%\tableofcontents % Print the table of contents

%\listoffigures % Print the list of figures

%\listoftables % Print the list of tables


%----------------------------------------------------------------------------------------
%	INTRODUCTION
%-------------------

\section{Introduction}
This document presents an overwiew of the work presented in my 2nd exam paper, along with a recommended reading list of papers on the topic.



%----------------------------------------------------------------------------------------
%	ABSTRACT
%----------------------------------------------------------------------------------------

\section*{Abstract} % This section will not appear in the table of contents due to the star (\section*)
Even before Artificial Intelligence was its own field of computational science, men have tried to mimic the activity of the human brain. In the early 1940s the first artificial neuron models were created as purely mathematical concepts. Over the years, ideas from neuroscience and computer science were used to develop the modern Neural Network. The interest in these models rose quickly but fell when they failed to be successfully applied to practical applications, and rose again in the late 2000s with the drastic increase in computing power, notably in the field of natural language processing, for example with the state-of-the-art speech recognizer making heavy use of deep neural networks.

Recurrent Neural Networks (RNNs), a class of neural networks with cycles in the network, exacerbates the difficulties of traditional neural nets. Slow convergence limiting the use to small networks, and difficulty to train through gradient-descent methods because of the recurrent dynamics have hindered research on RNNs, yet their biological plausibility and their capability to model dynamical systems over simple functions makes then interesting for computational researchers. 

Reservoir Computing emerges as a solution to these problems that RNNs traditionally face. Promising to be both theoretically sound and computationally fast, Reservoir Computing has already been applied successfully to numerous fields: natural language processing, computational biology and neuroscience, robotics, even physics. The survey paper will explore the history and appeal of both traditional feed-forward and recurrent neural networks, before describing the theory and models of this new reservoir computing paradigm. Finally recent papers using reservoir computing in a variety of scientific fields will be reviewed.

%----------------------------------------------------------------------------------------
%	AUTHOR AFFILIATIONS
%----------------------------------------------------------------------------------------

{\let\thefootnote\relax\footnotetext{* \textit{Department of Computer Science, the Graduate Center, CUNY, New York}}}

%{\let\thefootnote\relax\footnotetext{\textsuperscript{1} \textit{Department of Chemistry, University of Examples, London, United Kingdom}}}

%----------------------------------------------------------------------------------------

\newpage % Start the article content on the second page, remove this if you have a longer abstract that goes onto the second page
\enlargethispage*{\baselineskip}

%----------------------------------------------------------------------------------------
%	READING LIST
%----------------------------------------------------------------------------------------

\section{Recommended Reading List}

\subsection{Historical Papers}
P. Werbos, "Beyond regression: New tools for prediction and analysis in the behavioral sciences." (1974).\\

\noindent
P.Werbos, "Backpropagation through time: what it does and how to do it." Proceedings of the IEEE 78.10 (1990): 1550-1560.

\subsection{Reservoir Computing Overview}
M. Luko{\v{s}}evi{\v{c}}ius, and J. Herbert, "Reservoir computing approaches to recurrent neural network training." Computer Science Review 3.3 (2009): 127-149.

\subsection{Reservoir Computing Applications}
F. Triefenbach, A. Jalalvand, B. Schrauwen, and J.-P. Martens, “Phoneme recognition with large hierarchical reservoirs,” in Advances in Neural Information Processing Systems, Neural Information Processing System Foundation, (2010).\\

\noindent
L. Appeltant, M. C. Soriano, G. Van der Sande, J. Danckaert, S. Massar, J. Dambre,
B. Schrauwen, C. R. Mirasso, and I. Fischer, “Information processing using a single dynamical node as complex system,” Nature communications, (2011)\\

\noindent
A. Bernacchia, H. Seo, D. Lee, and X.-J. Wang, “A reservoir of time constants for memory traces in cortical neurons,” Nature Neuroscience, (2011).\\

\noindent
A. Jalalvand, “Connected digit recognition by means of reservoir computing,” Interspeech, (2011).\\

\noindent
X. Hinaut, “On-line processing of grammatical structure using reservoir computing,” Artificial Neural Networks and Machine Learning ICANN 2012, (2012).\\

\noindent
M. Oubbati, T. Oess, C. Fischer, and G. Palm, “Multiobjective reinforcement learning using adaptive dynamic programming and reservoir computing,” European Conference on Machine Learning and Principles and Practice of Knowledge Discovery in Databases, (2012).\\

\noindent
X. Hinaut, M. Petit, G. Pointeau, and P. F. Dominey, “Exploring the acquisition and production of grammatical constructions through human-robot interaction with echo state networks,” Frontiers in Neurorobotics, (2014).

\subsection{Related Works}
G.-B. Huang, D. H. Wang, and Y. Lan, “Extreme learning machines: a survey,” International Journal of Machine Learning and Cybernetics, (2011).\\

\vspace{-1mm}
\noindent
L. Appeltant, G. Van der Sande, J. Danckaert, and I. Fischer, “Constructing optimized binary masks for reservoir computing with delay systems,” Scientific reports, (2014).

\end{document}